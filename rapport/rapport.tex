\documentclass[a4paper, 12pt, oneside]{article}

\usepackage[utf8]{inputenc}
\usepackage[T1]{fontenc}
\usepackage[french]{babel}
\usepackage{array}
\usepackage{shortvrb}
\usepackage{listings}
\usepackage[fleqn]{amsmath}
\usepackage{amsfonts}
\usepackage{fullpage}
\usepackage{enumerate}
\usepackage{graphicx}
\usepackage{subfigure}
\usepackage{alltt}
\usepackage{url}
\usepackage{indentfirst}
\usepackage{eurosym}
\usepackage{listings}
\usepackage{titlesec, blindtext, color}
\usepackage[table,xcdraw,dvipsnames]{xcolor}
\usepackage[unicode]{hyperref}
\usepackage{url}
\usepackage{float}

\definecolor{mygray}{rgb}{0.5,0.5,0.5}

%%%% Page de garde %%%%

\title{\textbf{Introduction aux processus stochastiques}\\
	   Analyse de propagation d'un virus dans un réseau}
\author{Maxime GOFFART \\180521 \and Olivier JORIS\\182113}
\date{Année académique 2019 - 2020}

\begin{document}

\maketitle
\newpage

\tableofcontents
\newpage

\section{Introduction}

Les processus stochastiques permettent d'étudier des phénomènes aléatoires dans divers secteurs : l'économie, la climatologie, la météorologie, la biologie, \dotso

En particulier dans ce projet, il nous a été demandé d'étudier un phénomène d'actualité : la propagation d'un virus au sein d'un réseau, pouvant peut être modélisé à l'aide d'une chaîne de Markov. Ainsi, ce projet nous a permis d'appliquer les concepts vus au cours sur un exemple concret et d'actualité.

\section{Structure du programme} % je ne sais pas si on la met maintenant ou après

\section{Etude du modèle exact}

\subsection{Question 1}
	
	Le modèle proposé dans l'énoncé est bien un processus de Markov en temps discret caractérisé par ses $3^N$ états ($N$ étant la taille de la population). Les états de cette chaînes sont caractérisés par la suite de longueur $N$ des catégories (S, R ou I) auquel appartiennent les individus (les individus étant indexés de 1 à N) à l'instant t. Par exemple : pour $N = 3$, l'état "'S' 'I' 'I'" représente le fait que le premier individu est susceptible d'être infecté et que les deux derniers sont infectés.
	
	 Les probabilités de transitions d'un état à celui de l'instant suivant de la chaîne dépendent à la fois de chacune des catégories des individus à l'instant initial et de leurs interactions avec des personnes infectées (modélisées par le graphe $W$).
	
	Les états de cette chaîne qui sont uniquement composés d'individus de la catégorie 'R' ou\footnote{Il s'agit d'un "ou" inclusif.} de la catégorie 'S' sont absorbants car la propagation du virus n'est plus possible s'il n'y a plus d'infectés. Il en va de même pour les états dans lesquels les infectés n'ont de contact avec personne\footnote{Cela correspond à une ligne remplie de 0 dans le graphe $W$.}. Cette chaîne n'est donc ni irréductible, ni régulière, ni périodique.

\subsection{Question 2}

Il faudra en moyenne ... à un individu pour guérir une fois infecté.
	
\end{document}